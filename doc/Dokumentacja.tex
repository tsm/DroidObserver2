\documentclass[11pt,a4paper]{article}
\usepackage[utf8]{inputenc}
\usepackage[T1]{fontenc}
\usepackage[polish]{babel}
\usepackage{lmodern}
\usepackage{graphicx}
\usepackage{epstopdf}
\usepackage{anysize}
\usepackage{makeidx}
\usepackage{hyperref}



\makeatletter
\renewcommand{\maketitle}{
\begin{titlepage}
\begin{center}

\LARGE{AKADEMIA GÓRNICZO-HUTNICZA}

\vspace*{1cm}
\includegraphics[scale=1.8]{agh.eps}
\vspace*{1cm}

\LARGE{im. Stanisława Staszica w Krakowie}

\rule{\textwidth}{0.4mm}
\LARGE \textsc{\@title}
\rule{\textwidth}{0.4mm}

\vspace*{5mm}


\large
\emph{Autorzy:}\\
Tomasz \textsc{Czarnik}\\
Krzysztof \textsc{Śmiłek}\\

\vfill
\vspace*{\stretch{8}}
\rule{\textwidth}{0.4mm}

\large{Wydział Elektroniki, Automatyki, Informatyki i Elektrotechniki}\\
\large{Katedra Automatyki}\\
\vspace*{\stretch{7}}
\@date

\end{center}

\end{titlepage}
}
\makeatother

\title{DroidObserver}
\date{\today}

\makeindex

\begin{document}

\maketitle

\newpage

\tableofcontents

\newpage

\section{Wstęp}

\cite{2009UltraDeponti}
\cite{2010ComputingEttinger}
\cite{2011IEEEIntGoldman}
\cite{2011IEEERadioMitchell}
\cite{2009IEEEEngineeringSposaro}
\cite{2011DroidColunas}
\cite{2010IEEEEMBCDoukas}
\cite{2010IEEEEMBCSposaro}
\cite{2009IEEEBIBEWang}
\cite{2010NISSYang}
\cite{2010AndroidHashimi}

\section {Sformułowanie zadania projektowego}
\subsection {Obszar i przedmiot projektowania}
\subsubsection {Dziedzina problemu}
\subsubsection {Obszar modelowania}
\subsubsection {Zakres odpowiedzialności systemu}
\subsection {Zwięzła nazwa problemu}
\subsection {Cele do osiągnięcia}
\subsubsection {Cele rezultatu}
\subsubsection {Cele projektu}

\section {Opis wymagań}
\subsection {Funkcje systemu z punktu widzenia użytkownika}
\subsection {Przepływy informacyjne doprowadzone do i wyprowadzane z systemu}
\subsection {Sygnalizowane specjalne wymagania i ograniczenia}

\section {Analiza systemu– diagramy UML}
\subsection {Diagram przypadków użycia systemu}
\subsection {Diagram klas systemu}
\subsection {Diagram sekwencji}
\subsection {Diagram najważniejszego stanu systemu}
\subsection {Diagram Komunikacji systemu}
\subsection {Diagram Komponentów systemu}

\section {Opis zmian wprowadzonych w zrealizowanym Systemie }
\subsection {Zmiany na etapie projektowania klas}
\subsection {Zmiany w funkcjonowaniu systemu}

\section {Instrukcja obsługi Systemu}

\section{Bibliografia}

\bibliographystyle{plain}
\bibliography{bibliography}

\end{document}
