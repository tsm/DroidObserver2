\documentclass[11pt,a4paper]{article}
\usepackage[utf8]{inputenc}
\usepackage[T1]{fontenc}
\usepackage[polish]{babel}
\usepackage{lmodern}
\usepackage{graphicx}
\usepackage{epstopdf}
\usepackage{anysize}
\usepackage{makeidx}
\usepackage{hyperref}



\makeatletter
\renewcommand{\maketitle}{
\begin{titlepage}
\begin{center}

\LARGE{AKADEMIA GÓRNICZO-HUTNICZA}

\vspace*{1cm}
\includegraphics[scale=1.8]{agh.eps}
\vspace*{1cm}

\LARGE{im. Stanisława Staszica w Krakowie}

\rule{\textwidth}{0.4mm}
\LARGE \textsc{\@title}
\rule{\textwidth}{0.4mm}

\vspace*{5mm}


\large
\emph{Autorzy:}\\
Tomasz \textsc{Czarnik}\\
Krzysztof \textsc{Śmiłek}\\

\vfill
\vspace*{\stretch{8}}
\rule{\textwidth}{0.4mm}

\large{Wydział Elektroniki, Automatyki, Informatyki i Elektrotechniki}\\
\large{Katedra Automatyki}\\
\vspace*{\stretch{7}}
\@date

\end{center}

\end{titlepage}
}
\makeatother

\title{DroidObserver}
\date{\today}

\makeindex

\begin{document}

\maketitle

\newpage

\tableofcontents

\newpage

\section{Wstęp}

\cite{2009UltraDeponti}
\cite{2010ComputingEttinger}
\cite{2011IEEEIntGoldman}
\cite{2011IEEERadioMitchell}
\cite{2009IEEEEngineeringSposaro}
\cite{2011DroidColunas}
\cite{2010IEEEEMBCDoukas}
\cite{2010IEEEEMBCSposaro}
\cite{2009IEEEBIBEWang}
\cite{2010NISSYang}
\cite{2010AndroidHashimi}

\section {Sformułowanie zadania projektowego}
\subsection {Obszar i przedmiot projektowania}
\subsubsection {Dziedzina problemu}
\subsubsection {Obszar modelowania}
\subsubsection {Zakres odpowiedzialności systemu}
\subsection {Zwięzła nazwa problemu}
\subsection {Cele do osiągnięcia}
\subsubsection {Cele rezultatu}
\subsubsection {Cele projektu}

\section {Opis wymagań}
\subsection {Funkcje systemu z punktu widzenia użytkownika}
\subsection {Przepływy informacyjne doprowadzone do i wyprowadzane z systemu}
\subsection {Sygnalizowane specjalne wymagania i ograniczenia}

\section {Analiza systemu– diagramy UML}
\subsection {Diagram przypadków użycia systemu}
\subsection {Diagram klas systemu}
\subsection {Diagram sekwencji}
\subsection {Diagram najważniejszego stanu systemu}
\subsection {Diagram Komunikacji systemu}
\subsection {Diagram Komponentów systemu}

\section {Opis zmian wprowadzonych w zrealizowanym Systemie }
\subsection {Zmiany na etapie projektowania klas}
\subsection {Zmiany w funkcjonowaniu systemu}

\newpage
\section {Instrukcja obsługi Systemu}
\subsection{Instrukcja instalacji}

\subsubsection{Instalacja aplikacji na telefonie}
Plik DroidObserver.apk wgrać na telefon z systemem Android 2.2 (lub wyższym) i uruchomić go.
Instalacja powinna rozpocząć się automatycznie.

\subsubsection{Instalacja bazy danych}
\begin{enumerate}
\item Założyc nową bazę danych
\item W nowo stworzonej bazie danych uruchomić skrypt droidobserver.sql (plik ten znajduje się w folderze php) i zaobserwować czy powstały nowe tabele
\end{enumerate}

\subsubsection{Instalacja serwera}
\begin{enumerate}
\item Otworzyć plik config.php w edytorze tekstu
\item Zmienić natępujące parametry według wzoru:
\subitem \$baza = 'adres\_bazy\_danych';  (np. mysql.agh.edu.pl:3306);
\subitem \$login = 'login\_do\_bazy\_danych'; 
\subitem \$haslo = 'hasło\_do\_bazy\_danych';
\subitem \$database\_name = 'nazwa\_bazy\_danych'
\subitem \$admin\_login = 'login\_administratora'
\subitem \$admin\_haslo = 'haslo\_administratora'
\item Do katalogu public\_html (na serwerze) przegrać całą zawartośc folderu php
\end{enumerate}

\subsection{Instrukcja obsługi systemu}
\subsubsection{Instrukcja obsługi serwisu WWW}

Serwis WWW jest dostepny pod adresem:
\begin{itemize}
\item www.adresSerwera.domena - dla zwykłego użytkownika 
\item www.adresSerwera.domena/admin.php - dla administratora systemu
\end{itemize}

{\bf Panel użytkownika}\\
Aby skorzystać z panelu użytkownika należy zalogować się do systemu, poprzez podanie odpowiedniego loginu i hasła.
Aby uzyskać dane dostępowe należy skontaktować się z admnistratorem systemu.
Po zalogowaniu się użytkownik ma dostęp do nastepujących elementów:
\begin{itemize}
\item Przeglądanie aktualności (nie wymaga logowania się)
\item Sprawdzanie swoich danych osobowych (Imię, Nazwisko, telefon, email, nazwa choroby)
\item Odczytywanie ostatnich tras GPS 
\subitem Po kliknięciu w daną datę otwiera się okno z wszystkimi trasami z danego dnia
\item Przeglądanie zdjęć wysłanych na serwer
\end{itemize}

{\bf Panel administratora}\\
Panel administratora poza standartowymi funkcjonalnościami umożliwia także:
\begin{itemize}
\item Dodawanie aktualności
\item Dodawanie nowych pacjentów / Przeglądanie bazy wszystkich pacjentów
\item Odczytywanie trad GPS każdego z pacjentów
\item Przeglądanie zdjęć wyslanych przez konkretnego pacjenta
\end{itemize}

\subsubsection{Instrukcja obsługi aplikacji}

Po otwarciu nowo zainstalowanej aplikacji wyświetlony zostaje monit o podanie odpowiednich ustawień.
Ich zmiany można dokonać także później poprzez wybranie z menu głównego opcji 'Ustawienia'.\\
\\
{\bf Widok opcji}
\begin{enumerate}
\item {\it Adres serwera*} - niezbędny do uzyskania połączenia (http://adresSerwera.domena/)
\item {\it Login*} - unikalny login pacjenta
\item {\it Hasło*} - używane do logowania na serwerze
\item {\it Telefon alarmowy*} - Używany do połączen alarmowych
\item Częstotliwośc odświeżania - Interwał pomiędzy kolejnymi wysyłanymi współrzędnymi
\item Rozdzielczość aparatu - Rozdzielczość w jakiej robione będą zdjęcia
\end{enumerate}

* - pola unikalne (pozostałe opcje posiadają wartości domyślne)
\\\\
Po ustawieniu odpowiednich opcji nastepuje powrót do ekranu głównego. \\Można z niego wykonać jedną z poniższych opcji:
\begin{itemize}
\item Skontaktować się natychmiastowo z numerem alarmowym poprzez kliknięcie w duży czerwony przycisk na środku ekranu.
\item Zrobic i wysłać zdjęcie na serwer  poprzez wybór odpowiedniej opcji z menu.
\item Aktywować usługę wysyłania położenia GPS poprzez wybór odpowiedniej opcji z menu.
\item Wyłączyć usługę wysyłanie położenia GPS  poprzez wybór odpowiedniej opcji z menu.
\item Wyjść z programu poprzez wybór odpowiedniej opcji z menu.
\end{itemize}

\vspace{10pt}
Aplikacja jest skonstruowana tak, by automatyzować pewne procesy. W szczególności umożliwia ona zdalne włączanie/wyłączanie odpowiednich opcji programu, poprzez analizę przychodzących SMSów. Jeśli aplikacja napotka na SMS z konkretną treścią uruchamia ona jedną z opcji programu. Oto lista obsługiwanych komend:
\begin{enumerate}
\item GetGPS - wysyła pod numer alarmowy SMSa z aktualną lokalizacją 
\item StartGPS - uruchamia usługę wysyłania położenia GPS
\item StopGPS - zatrzymuje usługę wysyłania położenia GPS
\item TakePhoto - robi zdjęcie i wysyła na serwer
\end{enumerate}
\newpage
\section{Bibliografia}

\bibliographystyle{plain}
\bibliography{bibliography}

\end{document}
