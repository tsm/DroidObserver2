\documentclass[11pt,a4paper]{article}
\usepackage[utf8]{inputenc}
\usepackage[T1]{fontenc}
\usepackage[polish]{babel}
\usepackage{lmodern}
\usepackage{graphicx}
\usepackage{epstopdf}
\usepackage{anysize}
\usepackage{makeidx}
\usepackage{hyperref}


\makeatletter
\renewcommand{\maketitle}{
\begin{titlepage}
\begin{center}

\LARGE{AKADEMIA GÓRNICZO-HUTNICZA}

\vspace*{1cm}
\includegraphics[scale=1.8]{agh.eps}
\vspace*{1cm}

\LARGE{im. Stanisława Staszica w Krakowie}

\rule{\textwidth}{0.4mm}
\LARGE \textsc{\@title}
\rule{\textwidth}{0.4mm}

\vspace*{5mm}


\large
\emph{Autorzy:}\\
Tomasz \textsc{Czarnik}\\
Krzysztof \textsc{Śmiłek}\\

\vfill
\vspace*{\stretch{8}}
\rule{\textwidth}{0.4mm}

\large{Wydział Elektroniki, Automatyki, Informatyki i Elektrotechniki}\\
\large{Katedra Automatyki}\\
\vspace*{\stretch{7}}
\@date

\end{center}

\end{titlepage}
}
\makeatother

\title{Projekt i implementacja mobilnego systemu monitorowania pacjenta\\DroidObserver}
\date{\today}

\makeindex

\begin{document}

\maketitle

\newpage

\tableofcontents

\newpage

\section {Sformułowanie zadania projektowego}
\subsection {Obszar i przedmiot projektowania}
\subsubsection {Dziedzina problemu}
Niewielkie upośledzenie pamięci, wyrażające się zapominaniem pewnych stosunkowo niedawnych faktów, takich jak wydarzenia sprzed godziny lub z dnia poprzedniego, jest stosunkowo częstą dolegliwością wieku starczego.
Nie budzi to niepokoju otoczenia, nawet gdy nasilenie tych zaburzeń jest nieco większe, niż u innych osób w tym samym wieku. W powszechnym przekonaniu upośledzenie pamięci jest bowiem typową cechą starości, wynikającą z rozwijającej się nieuchronnie miażdżycy. Problem ten jeszcze do niedawna był bagatelizowany. Dopiero rozwój wiedzy na temat choroby Alzheimera uświadomił wielu osobom, także lekarzom, że występujące w starości kłopoty z pamięcią są również wynikiem zmian chorobowych, a nie tylko prostego procesu starzenia się.\\
\\
 Wywołane chorobą zaburzenia funkcji intelektualnej, takich jak pamięć, orientacja i myślenie, mogą być niewielkie i utrzymywać się na stale jednakowym poziomie, mogą też być wyraźnie nasilone i szybko postępujące. W tym pierwszym przypadku zaburzenia te ograniczają się tylko do pamięci, w drugim zaburzeniom pamięci towarzyszą z reguły również inne, postępujące zaburzenia funkcji intelektualnych. Powoduje to wyraźne problemy w życiu społecznym chorych. Nie mogą oni kontynuować dotychczasowej pracy zawodowej, a nawet poprawnie funkcjonować w domu. Stan ten nazywany jest otępieniem (demencją). Określenie to często bywa nadużywane. Według powszechnie przyjętych kryteriów, otępieniem nazywamy obniżenie funkcji intelektualnych człowieka w odniesieniu do ich poprzedniego, przedchorobowego poziomu, które powoduje problemy w życiu codziennym chorego. To obniżenie funkcji intelektualnych w otępieniu nie ogranicza się tylko do pamięci. Musi dotyczyć przynajmniej jeszcze jednej funkcji: orientacji, myślenia lub osądu.\\
\\
 Najczęstszymi przyczynami zaburzeń pamięci i intelektu są choroby zwyrodnieniowe i naczyniowe układu nerwowego. Mało nasilony proces zwyrodnieniowy, niewiele odbiegający od normalnego starzenia się, może spowodować tzw. zaburzenia pamięci związane z wiekiem, zwane też łagodną niepamięcią starczą lub łagodnym zaburzeniem funkcji poznawczych.\\
\\
 Najczęstszym rodzajem otępienia pochodzenia zwyrodnieniowego jest choroba Alzheimera. Znacznie rzadziej są to inne choroby, takie jak choroba Picka lub otępienie czołowe. Najczęstszym „naczyniowym” powodem otępienia są liczne, drobne zawały mózgu (otępienie wielozawałowe). Ponieważ niektóre części mózgu, np. wewnętrzna część płata skroniowego, są szczególnie ważne dla funkcji pamięci, ich uszkodzenie w wyniku pojedynczego zawału mózgu także może być powodem otępienia.\\
\\
 Ludzie dotknięci wyżej wymienionymi problemami mogą mieć problem w swobodnym poruszaniu się poza domem czy jednostką kliniczną. Może się zdarzyć, iż taka osoba bez bezpośredniego nadzoru opiekuna zgubi się i zapomni w jaki sposób wrócić do bezpiecznej lokalizacji. W tym wypadku z pomocą przychodzi nasza aplikacja. Pacjent może uzyskać pomoc naciskając tylko jeden czerwony przycisk znajdujący się na środku ekranu. Po nawiązaniu połączenia z opiekunem, może zostać poproszony o wycelowanie telefonu w kierunku charakterystycznych obiektów takich jak budynek, przystanek autobusowy czy rzeźba terenu. Opiekun lub lekarz natomiast mogą w prosty sposób monitorować miejsce pobytu pacjenta. 

\subsubsection {Obszar modelowania}

Projekt można podzielić na trzy współpracujące ze sobą części:
\begin{itemize}
\item Baza danych - przechowuje dane o pacjentach, kolekcjonuje lokalizacje GPS oraz ścieżki dostępu do plików.
\item Aplikacja webowa - pozwala na modyfikację, dodawanie oraz przeglądanie danych.
\item Aplikacja kliencką - zainstalowana na telefonie pacjenta reaguje na sygnały zewnętrzne (akcje klienta bądź żądania ze strony opiekuna bądź lekarza) i zajmuje się pozyskiwaniem i wysyłaniem danych na serwer.
\end{itemize}

Ze względu na strukturę organizacyjną wyróżniamy następujące funkcje:
\begin{itemize}
\item Administrator systemu - zarządza bazą danych pacjentów, w ogólnym wypadku jest to lekarz posiadający kartotekę pacjentów.
\item Opiekun - osoba, która ma dostęp do hasła pacjenta w serwisie, jego numer może być przypisany jako numer alarmowy w aplikacji klienckiej. Opiekun jest opcjonalny jeśli lekarz-administrator przejmie jego funkcje.
\item Pacjent - osoba dotknięta dysfunkcją, która wyraziła zgodę na dobrowolny udział w projekcie. Powinna zawsze nosić przy sobie urządzenie z systemem Android z zainstalowaną aplikacją kliencką.
\end{itemize}
\subsubsection {Zakres odpowiedzialności systemu}
W zakres odpowiedzialności realizowanego systemu wchodzą następujące obszary aktywności:
\begin{itemize}
\item Administracja danymi pacjenta.
\item Reakcja na sygnały zewnętrzne aplikacji klienckiej.
\item Wysyłanie współrzędnych geograficznych w oparciu o odczyt GPS.
\item Zdalne wykonanie zdjęcia i przesłanie na serwer.
\item Udostępnienie interfejsu dla opiekuna/pacjenta w celu dostępu do zebranych informacji.
\end{itemize}
Dodatkowo realizowane są następujące obszary aktywności:
\begin{itemize}
\item System newsów w celu informowania opiekunów/pacjentów o aktualnościach i zmianach w działaniu systemu.
\item Czerwony przycisk pomocy umożliwiający nawiązanie błyskawicznego kontaktu z numerem alarmowym.
\item System sprawdzania poprawności konfiguracji i pomoc w niej.
\end{itemize}


\subsection {Zwięzła nazwa problemu}
Aplikacja na system Android wspomagająca opiekę nad ludźmi dotkniętymi demencją, potrafiąca zdalnie określić położenie pacjenta (dzięki nadajnikowi GPS w telefonie) oraz wykonać zdjęcie aparatem. System zarządzany jest z poziomu przeglądarki internetowej lekarza (administratora), który może mieć pod opieką więcej niż jednego pacjenta.
\subsection {Cele do osiągnięcia}
Celem projektu jest stworzenie systemu wspomagającego opiekę nad ludźmi dotkniętymi demencją poprzez monitorowanie na żądanie obecnego położenia geograficznego pacjenta oraz możliwość wykonania zdjęcia. System powinien być jak najbardziej niezawodny w zróżnicowanych warunkach. Ze względu na to iż pacjentami w większości będą osoby starsze bez zbytniego obycia w technologiach mobilnych system powinien mieć niezwykle prosty oraz ograniczony do minimum interfejs użytkownika, a większość czynności powinna być wykonywana automatycznie. Jako, że w systemie przechowywane są prywatne dane pacjentów powinien zapewniać bezpieczeństwo tych danych i zminimalizowanie nadużyć.\\
\\
Dodatkowym celem było zapoznanie się za platformą Android oraz z metodyką projektowania systemów informatycznych w dziedzinie telemedycyny.
\section {Opis wymagań}
\subsection {Funkcje systemu z punktu widzenia użytkownika}
Aplikacja webowa:
\begin{enumerate}
  \item Przegląd aktualności (nie wymaga logowania)
  \item Informacje o pacjencie
  \item Mapy współrzędnych GPS
  \item Przegląd nadesłanych zdjęć
\end{enumerate}
Aplikacja kliencka:
\begin{enumerate}
\item Konfiguracja
\item Czerwony przycisk telefonu alarmowego
\item Wysłanie zdjęcia
\item Wysłanie współrzędnych GPS
\item Informowanie o wysyłaniu danych
\end{enumerate}
\newpage
\subsection {Przepływy informacyjne doprowadzone do i wyprowadzane z systemu}
Przepływ informacji przedstawia poniższy diagram:\\
  \begin{figure}[h]
    \includegraphics[scale=0.5]{data_flow.png}
    \caption{Przepływ informacji}
  \end{figure}
\\Przy czym należy pamiętać, iż komendy wydawać można także z telefonu Opiekuna (nie zostały one umieszczone na diagramie aby nie pogorszyć jego czytelności).
\subsection {Sygnalizowane specjalne wymagania i ograniczenia}
Zanim w ogóle rozpocznie się korzystanie z systemu należy podjąć kroki prawne regulujące zgodę pacjenta (bądź w przypadku niesamodzielności podmiot odpowiedzialny) na przetwarzanie jego prywatnych danych w szczególności danych osobowych oraz położenia geograficznego. Aplikacja kliencka napisana została w Android API 8, a więc powinna poprawnie funkcjonować na urządzeniach działających pod kontrolą systemu operacyjnego Android 2.2 bądź kompatybilnych nowszych. Niezbędne do wysyłania informacji jest połączenie z Internetem, zalecamy wykupienie miesięcznego pakietu danych z nielimitowanym transferem (prędkość przesyłania danych nie jest specjalnie istotna), w przypadku chwilowego braku połączenia internetowego aplikacja stara się kolekcjonować dana w celu wznowienia ich wysłania kiedy będzie to możliwe. Urządzenie musi posiadać sprawny moduł GPS, najlepiej z opcją A-GPS pozwalającą na szybsze odnajdywanie pozycji satelitarnej.\\
\\
Dla potrzeb serwera wymagany jest serwer wspierający standard PHP 4 oraz baza danych MySQL bądź PostgreSQL. W celu dostępu do serwisu wystarczy dowolna przeglądarka internetowa, zalecamy użycie najnowszej wersji Chrome.

\section {Analiza systemu– diagramy UML}
\subsection {Diagram przypadków użycia systemu}
\begin{figure}[h]
    \includegraphics[scale=0.66]{use_case.png}
    \caption{Diagram przypadków użycia}
 \end{figure}
Na diagramie przypadki użycia po stronie serwera zaznaczono błękitnym kolorem, natomiast po stronie aplikacji klienckiej zielonym.

\newpage
\subsection {Diagram klas systemu}
  \begin{figure}[h]
    \includegraphics[scale=0.4]{class_diagram.png}
    \caption{Diagram klas systemu}
  \end{figure}
Na rysunku przedstawiony został diagram klas systemu. Poszczególne kolory klas oznaczają:
\begin{itemize}
\item zielony - aktywność startowa (główna aktywność uruchamiana wraz ze startem aplikacji)
\item błękitny - usługi: nie mają interfejsu, działają w tle
\item pomarańczowy - zwykłe aktywności, posiadają interfejs
\item szary - dialog, wyświetla małe szare okienko z informacją
\item granatowy - aktywność preferencji
\end{itemize}
\subsection {Diagram sekwencji}
\begin{figure}[h]
    \includegraphics[scale=0.7]{sequence.png}
    \caption{Diagram sekwencji}
 \end{figure}
Kolory analogiczne do diagramu klas.
\newpage
\subsection {Diagram najważniejszego stanu systemu}
\begin{figure}[h]
    \includegraphics[scale=0.71]{state.png}
    \caption{Diagram stanu aplikacji DroidObsever}
 \end{figure}
\subsection {Diagram Komponentów systemu}
\begin{figure}[h]
    \includegraphics[scale=0.6]{component.png}
    \caption{Diagram komponentów}
 \end{figure}

\newpage
\section {Instrukcja obsługi Systemu}
\subsection{Instrukcja instalacji}

\subsubsection{Instalacja aplikacji na telefonie}
Plik DroidObserver.apk wgrać na telefon z systemem Android 2.2 (lub wyższym) i uruchomić go.
Instalacja powinna rozpocząć się automatycznie.

\subsubsection{Instalacja bazy danych}
\begin{enumerate}
\item Założyc nową bazę danych
\item W nowo stworzonej bazie danych uruchomić skrypt droidobserver.sql (plik ten znajduje się w folderze php) i zaobserwować czy powstały nowe tabele
\end{enumerate}

\subsubsection{Instalacja serwera}
\begin{enumerate}
\item Otworzyć plik config.php w edytorze tekstu
\item Zmienić natępujące parametry według wzoru:
\subitem \$baza = 'adres\_bazy\_danych';  (np. mysql.agh.edu.pl:3306);
\subitem \$logindb = 'login\_do\_bazy\_danych'; 
\subitem \$haslo = 'hasło\_do\_bazy\_danych';
\subitem \$database\_name = 'nazwa\_bazy\_danych'
\subitem \$admin\_login = 'login\_administratora'
\subitem \$admin\_haslo = 'haslo\_administratora'
\item Do katalogu public\_html (na serwerze) przegrać całą zawartośc folderu php
\end{enumerate}

\subsection{Instrukcja obsługi systemu}
\subsubsection{Instrukcja obsługi serwisu WWW}

Serwis WWW jest dostepny pod adresem:
\begin{itemize}
\item www.adresSerwera.domena - dla zwykłego użytkownika 
\item www.adresSerwera.domena/admin.php - dla administratora systemu
\end{itemize}

{\bf Panel użytkownika}\\
Aby skorzystać z panelu użytkownika należy zalogować się do systemu, poprzez podanie odpowiedniego loginu i hasła.
Aby uzyskać dane dostępowe należy skontaktować się z admnistratorem systemu.
Po zalogowaniu się użytkownik ma dostęp do nastepujących elementów:
\begin{itemize}
\item Przeglądanie aktualności (nie wymaga logowania się)
\item Sprawdzanie swoich danych osobowych (Imię, Nazwisko, telefon, email, nazwa choroby)
\item Odczytywanie ostatnich tras GPS 
\subitem Po kliknięciu w daną datę otwiera się okno z wszystkimi trasami z danego dnia
\item Przeglądanie zdjęć wysłanych na serwer
\end{itemize}

{\bf Panel administratora}\\
Panel administratora poza standartowymi funkcjonalnościami umożliwia także:
\begin{itemize}
\item Dodawanie aktualności
\item Dodawanie nowych pacjentów / Przeglądanie bazy wszystkich pacjentów
\item Odczytywanie tras GPS każdego z pacjentów
\item Przeglądanie zdjęć wyslanych przez konkretnego pacjenta
\end{itemize}
\vspace {20pt}
\begin{figure}[h]
\centering
\includegraphics[scale=0.7]{route.png}   
\caption {Wygląd trasy z panelu administratora}
\end{figure}
\vspace {20pt}

\subsubsection{Instrukcja obsługi aplikacji}

Po otwarciu nowo zainstalowanej aplikacji wyświetlony zostaje monit o podanie odpowiednich ustawień.
Ich zmiany można dokonać także później poprzez wybranie z menu głównego opcji 'Ustawienia'.\\
\\\\
{\bf Widok opcji}
\begin{enumerate}
\item {\it Adres serwera*} - niezbędny do uzyskania połączenia (http://adresSerwera.domena/)
\item {\it Login*} - unikalny login pacjenta
\item {\it Hasło*} - używane do logowania na serwerze
\item {\it Telefon alarmowy*} - Używany do połączen alarmowych
\item Częstotliwośc odświeżania - Interwał pomiędzy kolejnymi wysyłanymi współrzędnymi
\item Rozdzielczość aparatu - Rozdzielczość w jakiej robione będą zdjęcia
\end{enumerate}

* - pola unikalne (pozostałe opcje posiadają wartości domyślne)\\

\includegraphics[scale=0.9]{screen_settings.png}    
\includegraphics[scale=0.9]{screen_main.png}\\
\\\\
Po ustawieniu odpowiednich opcji nastepuje powrót do ekranu głównego. \\Można z niego wykonać jedną z poniższych opcji:
\begin{itemize}
\item Skontaktować się natychmiastowo z numerem alarmowym poprzez kliknięcie w duży czerwony przycisk na środku ekranu.
\item Zrobic i wysłać zdjęcie na serwer  poprzez wybór odpowiedniej opcji z menu.
\item Aktywować usługę wysyłania położenia GPS poprzez wybór odpowiedniej opcji z menu.
\item Wyłączyć usługę wysyłanie położenia GPS  poprzez wybór odpowiedniej opcji z menu.
\item Wyjść z programu poprzez wybór odpowiedniej opcji z menu.
\end{itemize}

\vspace{10pt}
Aplikacja jest skonstruowana tak, by automatyzować pewne procesy. W szczególności umożliwia ona zdalne włączanie/wyłączanie odpowiednich opcji programu, poprzez analizę przychodzących SMSów. Jeśli aplikacja napotka na SMS z konkretną treścią uruchamia ona jedną z opcji programu. Oto lista obsługiwanych komend:
\begin{enumerate}
\item GetGPS - wysyła pod numer alarmowy SMSa z aktualną lokalizacją 
\item StartGPS - uruchamia usługę wysyłania położenia GPS
\item StopGPS - zatrzymuje usługę wysyłania położenia GPS
\item TakePhoto - robi zdjęcie i wysyła na serwer
\end{enumerate}
\newpage
\section{Bibliografia}
Przy realizacji naszego projektu bardzo pomocne okazały się materiały dostarczone przez platformę IEEE
\cite{2009UltraDeponti}
\cite{2010ComputingEttinger}
\cite{2011IEEEIntGoldman}
\cite{2011IEEERadioMitchell}
\cite{2009IEEEEngineeringSposaro}
\cite{2011DroidColunas}
\cite{2010IEEEEMBCDoukas}
\cite{2010IEEEEMBCSposaro}
\cite{2009IEEEBIBEWang}
\cite{2010NISSYang}.
Jako, że był to nasz pierwszy projekt napisany na platformę Android musieliśmy nauczyć się programowania pod ten system. W tym celu sięgnęliśmy po doskonały podręcznik 'Android 2. Tworzenie aplikacji'\cite{2010AndroidHashimi}.

\bibliographystyle{plain}
\bibliography{bibliography}

\end{document}
